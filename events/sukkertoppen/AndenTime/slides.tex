\documentclass[12pt,t]{beamer}
\usetheme[greyauthor, % Grå tekst forfatter som KU vil have
         unit=ics, % Ændre til NAT, KU, eller unit=ics (diku)
         dk, % Sprog
         %style=simple, % Vandmærke eller billede
         footstyle=low, % Fjern stor footer
         wmark, % vandmærke på hver side
         logoplace=left % Logo til venstre
         %,sidebar % makes sidebar
         ]{Frederiksberg}
% nat for Science, ku for generic or unit=ics for DIKU
% Tilføj style=simple for vandmærke
\usepackage{pslatex}        % pæn skrift
\usepackage[utf8]{inputenc} % Implementerer Unicode
\usepackage{algpseudocode}
\usepackage{algorithm}
\usepackage{color}
\usepackage{minted}

\title{Espergærde gymnasium}
\subtitle{Python intro}
\author{
        Gymnasietjenesten på DIKU
}

\date[]{\today}


\begin{document}

\frame[plain]{\titlepage}
 \frame{\tableofcontents}

 \section{Introduktion til python}
     \begin{frame}[t]{Hvorfor python?}
         \begin{description}
             \item[Høj niveaus sprog] Man skal ikke tænke på hvordan
             maskinen fortolker det.
             \pause
             \item[Alsidigt] Bliver brugt mange steder,
             fra webudvikling til kræftforskning
             \pause
             \item[Nemt at lære] Det har en simpel pæn syntax og er meget
             tilgivende. Minder meget om pseudokode
         \end{description}
         \pause
         \begin{block}{Minimums algoritmen}
               \inputminted{python}{min.py}
        \end{block}
     \end{frame}


    \begin{frame}[t]{Forskell på sprog}
        \begin{block}{If statements}
            Hvis $a$ så gør $b$. \\
            Hvis $a$ så gør $b$ ellers gør $c$.
        \end{block}
        \begin{block}{Løkker}
            \begin{itemize}
                \item for i in range(0,10):
                \pause \item for i in lst:
                \pause \item while(point $<$ 100):
            \end{itemize}
    \end{block}
    \end{frame}


    \section{Øvelser}
    \begin{frame}[t]{Øvelsestid}
        Kode øvelser i python!

        \begin{block}{Urls}
            \begin{itemize}
                \item rotendahl.dk/espergaerde
                \item Kode : 1337
            \end{itemize}
        \end{block}
    \end{frame}




    %--- Next Frame ---%

\end{document}
