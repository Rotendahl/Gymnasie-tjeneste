\documentclass[11pt]{article}
\usepackage[a4paper, hmargin={2.8cm, 2.8cm}, vmargin={2.5cm, 2.5cm}]{geometry}
\usepackage{eso-pic} % \AddToShipoutPicture
\usepackage{graphicx} % \includegraphics

% Pakker til skrifttyper, tekst osv.
% %%%%%%%%%%%%%%%%%%%%%%%%%%%%%%%%%%%%%%%%%%%%%%%%%%%%%%%%%%%%%%%%%%%%%%%%%%%%%
    \usepackage[utf8]{inputenc} % Implementere Unicode
    \usepackage[T1]{fontenc}    % Unicode skrifttype, fx. é skrives som 1 tegn
 %    \usepackage[english]{babel} % Engelsk Ordbog
   \usepackage[danish]{babel}  % Dansk Ordbog
    \usepackage{microtype}      % Forbedre linjeombrydningen
    \usepackage{libertine}      % Skrifttype
    \usepackage[scaled=0.83]{inconsolata} % Skrifttype til kode til kode
% %%%%%%%%%%%%%%%%%%%%%%%%%%%%%%%%%%%%%%%%%%%%%%%%%%%%%%%%%%%%%%%%%%%%%%%%%%%%%


% Pakker til matematik og kode.
% %%%%%%%%%%%%%%%%%%%%%%%%%%%%%%%%%%%%%%%%%%%%%%%%%%%%%%%%%%%%%%%%%%%%%%%%%%%%%
    \usepackage{mathtools}       % Udvidelse til amsmath pakken
    \usepackage{algpseudocode}   % pseudocode til algoritmer
    \usepackage{algorithm}       % Pakke til algoritmer
    \usepackage{amsthm}          % Pakke til Theroms
% %%%%%%%%%%%%%%%%%%%%%%%%%%%%%%%%%%%%%%%%%%%%%%%%%%%%%%%%%%%%%%%%%%%%%%%%%%%%%

% Pakker til layout.
% %%%%%%%%%%%%%%%%%%%%%%%%%%%%%%%%%%%%%%%%%%%%%%%%%%%%%%%%%%%%%%%%%%%%%%%%%%%%%
    \usepackage{fancyhdr}        % Gør det muligt at bruge sidehoveder
    \usepackage{graphicx}        % Mulighed for bl.a. \includegraphics
%    \usepackage[dvipsnames]{xcolor} % Mulighed for \definecolor
    \usepackage{colortbl}        % Hvis man vil farvelægge sine tabeller
    \usepackage{array}           % Gør miljøerne array og tabular lidt bedre
    \usepackage{parskip}         % Første paragraf i afsnit indrykkes ikke
    \usepackage{listings}        % Pakke til at indsætte kode
    \usepackage{enumitem}        % Gør det muligt at tilpasse lister
    \usepackage{titlesec}        % Tilpassing af afstand mellem sektioner
    \usepackage[lastpage,user]{zref} % Side x af y
% %%%%%%%%%%%%%%%%%%%%%%%%%%%%%%%%%%%%%%%%%%%%%%%%%%%%%%%%%%%%%%%%%%%%%%%%%%%%%


% Ikke standard pakker
% %%%%%%%%%%%%%%%%%%%%%%%%%%%%%%%%%%%%%%%%%%%%%%%%%%%%%%%%%%%%%%%%%%%%%%%%%%%%%
%    \usepackage{boxproof}        % Til at lave bevis kasser i logik
    %\usepackage{semantic}        % Pakke til logik med ->, |- osv.
%    \usepackage{daymonthyear}    % Giver info om dato.
% %%%%%%%%%%%%%%%%%%%%%%%%%%%%%%%%%%%%%%%%%%%%%%%%%%%%%%%%%%%%%%%%%%%%%%%%%%%%%


% Implementere en række makroer og de pakker der er importeret
% %%%%%%%%%%%%%%%%%%%%%%%%%%%%%%%%%%%%%%%%%%%%%%%%%%%%%%%%%%%%%%%%%%%%%%%%%%%%%
    \pagestyle{fancy}                        % Implementere sidehoved
    \lhead{Københavns Universitet} % Venstre sidehoved
    \rhead{Kulturnatten på DIKU}                      % Højre sidehoved
    \setlist{nolistsep}            % Formindsker mellemrum mellem listepunkter


    % Definitioner af farver
    % %%%%%%%%%%%%%%%%%%%%%%%%%%%%%%%%%%%%%%%%%%%%%%%%%%%%%%%%%%%%%%%%%%%%
        \definecolor{KURed1}{RGB}{144,26,30}    % Official KU Red 1
        \definecolor{KURed2}{RGB}{199,36,41}    % Unofficial KU Red
        \definecolor{KUGray1}{RGB}{102,102,102} % Official KU Gray 1
        \definecolor{KUGray2}{RGB}{133,133,133} % Official KU Gray 2
        \definecolor{KUGray3}{RGB}{163,163,163} % Official KU Gray 3
        \definecolor{KUGray4}{RGB}{194,194,194} % Official KU Gray 4
        \definecolor{KUGray5}{RGB}{224,224,224} % Official KU Gray 5
    % %%%%%%%%%%%%%%%%%%%%%%%%%%%%%%%%%%%%%%%%%%%%%%%%%%%%%%%%%%%%%%%%%%%%


    % Mindsker afstanden mellem sektioner
    % %%%%%%%%%%%%%%%%%%%%%%%%%%%%%%%%%%%%%%%%%%%%%%%%%%%%%%%%%%%%%%%%%%%%
        \titlespacing\section{0pt}{12pt plus 4pt minus 2pt}
                                  {0pt plus 1pt minus 3pt}
        \titlespacing\subsection{0pt}{12pt plus 4pt minus 2pt}
                                  {0pt plus 1pt minus 3pt}
        \titlespacing\subsubsection{0pt}{12pt plus 4pt minus 2pt}
                                  {0pt plus 1pt minus 3pt}
    % %%%%%%%%%%%%%%%%%%%%%%%%%%%%%%%%%%%%%%%%%%%%%%%%%%%%%%%%%%%%%%%%%%%%


    % Genveje
    % %%%%%%%%%%%%%%%%%%%%%%%%%%%%%%%%%%%%%%%%%%%%%%%%%%%%%%%%%%%%%%%%%%%%
    \def\meta#1{\mbox{$\langle\hbox{#1}\rangle$}} % Mbox til premiser
    \def\intro#1{{#1}{\cal I}}
    \def\elim#1{{#1}{\cal E}}
    \let\imp\to
    \let\implies\to

    % %%%%%%%%%%%%%%%%%%%%%%%%%%%%%%%%%%%%%%%%%%%%%%%%%%%%%%%%%%%%%%%%%%%%


    % Milø til kode
    % %%%%%%%%%%%%%%%%%%%%%%%%%%%%%%%%%%%%%%%%%%%%%%%%%%%%%%%%%%%%%%%%%%%%
        \lstdefinestyle{kode}{
            basicstyle=\scriptsize\ttfamily\color{black},
            keywordstyle=\bfseries\color{KURed1},
            commentstyle=\bfseries\color{KUGray1},
            identifierstyle=\color{black},
            stringstyle=\color{KURed2},
            breaklines=false,
            showspaces=false,
            showstringspaces=false,
            extendedchars=true,
            breakatwhitespace=false,
        }
        \lstnewenvironment{code}[1][]
        {\minipage{\linewidth}
            \lstset{
                #1,
                style=kode,
                escapeinside={!!}{!!},
                frame=tb,
                }
        }
        {\endminipage}
    % %%%%%%%%%%%%%%%%%%%%%%%%%%%%%%%%%%%%%%%%%%%%%%%%%%%%%%%%%%%%%%%%%%%%

% %%%%%%%%%%%%%%%%%%%%%%%%%%%%%%%%%%%%%%%%%%%%%%%%%%%%%%%%%%%%%%%%%%%%%%%%%%%%%


%%%%%%%%%%%%%%%%%%%%      Her starter documentet    %%%%%%%%%%%%%%%%%%%%%%%%%%%
\begin{document}
\pagenumbering{gobble}
%%%%%%%%%%%%%%%%%%%%%%%%%%%%%%%%%%%%%%%%%%%%%%%%%%%%%%%%%%%%%%%%%%%%%%%%%%%%%
\section{Introduktion}
    Tak for besøget her til kulturnatten på Datalogisk Institut ved Københanvs 
    Universitet (DIKU). Vi håber at du gik herfra med en masse ny viden! 
    Hvis du synes, at det lød spændende, har vi her samlet noget information. 

\section{DIKU}
\begin{description}
    \item[Besøg fra DIKU]~\\
        DIKU tilbyder besøg i din gymnasieklasse fra rejseholdet; enten med et 
        kort oplæg om de sociale og faglige aspekter ved uddannelsen, eller et 
        længerevarende undervisningsforløb. Hvis du er interesseret kan du få 
        din lærer til at skrive en mail til heidi.pa@di.ku.dk.
    \item[Studerende for en dag]~\\
        ``Studerende for en dag'' er et initiativ på SCIENCE, hvor 
        gymnasieelever og andre  universitetsaspiranter følger en studerende på
        et af de naturvidenskabelige fag gennem en hel dag. 
        Her får du altså mulighed for at se en helt almindelig hverdag for 
        en studerende på datalogi. 
        Hvis du er interesseret så kan du lave en aftale på 
        http://www.science.ku.dk/oplev-science/gymnasiet/studerende-for-en-dag/.
\end{description}

\section{Programmering}
    Vil du lære mere om programmering er der her nogle nyttige links.
    \begin{description}
        \item[Processing]~\\
            Tekstbaseret programmering, men med øjeblikkeligt visuel respons.
            En god introduktion til det at programmere.
            Undervisningsmaterialet er på dansk.\\
            http://da.khanacademy.org/computing/computer-programming/programming
        \item[Codecademy]~\\
            Hvis du har prøvet kræfter med en Processing, og du synes at det 
            var sjovt, så prøv Codecademy. Der er intuitive guides til en masse 
            forskellige programmeringssprog. \\ 
            https://www.codecademy.com/
        \item[Datalogi]~\\
            Harvard University har lavet et fantastisk introduktionskursus til 
            datalogi, som man kan følge i sit eget tempo, hvis man ellers er 
            engelskkyndig. Man lærer her noget datalogi som strækker sig ud over
            det at programmere. \\
            www.edx.org/course/introduction-computer-science-harvardx-cs50x
    \end{description}
\end{document}
