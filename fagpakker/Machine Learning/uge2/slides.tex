\documentclass[12pt,t]{beamer}
\usetheme[greyauthor, % Grå tekst forfatter som KU vil have
         unit=ics, % Ændre til NAT, KU, eller unit=ics (diku)
         dk, % Sprog
         %style=simple, % Vandmærke eller billede
         footstyle=low, % Fjern stor footer
         wmark, % vandmærke på hver side
         logoplace=left % Logo til venstre
         %,sidebar % makes sidebar
         ]{Frederiksberg}
% nat for Science, ku for generic or unit=ics for DIKU
% Tilføj style=simple for vandmærke
\usepackage{pslatex}        % pæn skrift
\usepackage[utf8]{inputenc} % Implementerer Unicode
\usepackage{algpseudocode}
\usepackage{algorithm}
\usepackage{color}
\usepackage{minted}

\title{Fagpakke dag 2}
\subtitle{Algoritmer og problemløsning}
\author{
        Arinbjörn Brandsson \\
        Benjamin Rotendahl  \\
        Mathias Mortensen
}

\date[]{\today}


\begin{document}

\frame[plain]{\titlepage}
 \frame{\tableofcontents}

\section{Program og Opsumering}

\begin{frame}
    \frametitle{Program for de kommende uger}
    \begin{block}{Uge 1}
        \begin{itemize}
            \item Introduktion til algoritmer.
            \item Algoritme design og metoder.
            \item Hvordan man kan sammenligne forskellige løsninger.
            \item Øvelser i algoritmer.
        \end{itemize}
    \end{block}
    \pause
    \begin{block}{Uge 2}
        \begin{itemize}
            \item \alert{Introduktion til Programmering.}
            \item \alert{Programmerings øvelser.}
        \end{itemize}
    \end{block}
    \pause
    \begin{block}{Uge 3}
        \begin{itemize}
            \item Databehandling med Machine Learning.
            \item Øvelser i dataanalyse.
        \end{itemize}
    \end{block}
\end{frame}


\section{Opsumering fra sidst}
    \begin{frame}[c]{Algoritmer og Køretid}
        \begin{block}{På dansk}
            En algoritme er en \alert{opskrift} på hvordan et bestemt problem
            kan løses.
        \end{block}
        \pause
        \begin{block}{Definition på køretid}
            En øvregrænse for den tid der bliver brugt på at løse et problem af
            størelse $n$. Skrives som
            $$
                O(n), O(n^2), O(n \lg n), O(n!), O\left( \frac{a}{b} \right)
            $$
        \end{block}
    \end{frame}

    \begin{frame}{Minimums algoritme}
        \transdissolve
        \begin{exampleblock}{Algoritme for minimums funktionen}
            Givet en liste $X = [x_1,x_2,\dots,x_n]$ ønsker vi at returnere det
            mindste tal i listen. Hvad er algoritmen og hvad er køretiden?
        \end{exampleblock}
        \begin{block}{Eksempel}
        \vspace{-1.5em}
        \begin{algorithm}[H]
            \caption{\newline Input: En liste $X=[x_1,x_2, \dots, x_n]$
                     \newline Ouput: Det mindste tal i listen.
            }
            \begin{algorithmic}
                \State min = $x_1$
                \For{$x_i$ in X}
                    \If{$x_i < min$}
                        \State min = $x_i$
                    \EndIf
                \EndFor
            \end{algorithmic}
        \end{algorithm}
        \end{block}
    \end{frame}
%
 \section{Introduktion til python}
     \frame{\tableofcontents[currentsection]}

     \begin{frame}[t]{Hvorfor python?}
         \begin{description}
             \item[Høj niveaus sprog] Man skal ikke tænke på hvordan
             maskinen fortolker det.
             \pause
             \item[Alsidigt] Bliver brugt mange steder,
             fra webudvikling til kræftforskning
             \pause
             \item[Nemt at lære] Det har en simpel pæn syntax og er meget
             tilgivende. Minder meget om pseudokode
         \end{description}
         \pause
         \begin{block}{Minimums algoritmen}
               \inputminted{python}{min.py}
        \end{block}
     \end{frame}

     \begin{frame}{Matematik og variabler}
         \vspace{-1em}
         \begin{block}{Matematik}
            \begin{itemize}
                \item 3 + 14 \pause
                \item 69 / 2 \pause
                \item 21 * 2 \pause
                \item 2 ** 2
            \end{itemize}
         \end{block}
         \pause
         \begin{block}{variabler}
             \begin{itemize}
                 \item pi  = 3.14
                \pause \item liv = 40 + 2
                \pause \item pi2 = pi * 2
                \pause \item fornavn = "Benjamin~"
                \pause \item efternavn = "Rotendahl"
                \pause \item mig = fornavn + efternavn
                \pause \item harKage = True
             \end{itemize}
         \end{block}
     \end{frame}

    \begin{frame}{Funktioner og lister}
        \begin{block}{lister}
            \begin{description}
                \item[lst = [1,2,3,4,5]] variablen ``lst'' er nu en liste med
                tal. \pause
                \item[nr = lst[0]] variablen nr er nu lig med det første tal i
                listen. \pause
                \item[lst.append(6)] listen er nu blevet en længere og det sidste
                element er $6$.
            \end{description}
        \end{block}
        \pause
        \begin{block}{funktioner}
            \begin{itemize}
                \item mindst = min(lst)
            \pause  \item forksel = max(lst) - min(lst)
            \pause  \item print "hejsa"
            \pause  \item prikprod(v, u)
            \end{itemize}
        \end{block}
    \end{frame}


    \begin{frame}[t]{kontrol udtryk}
        \begin{block}{If statements}
            Hvis $a$ så gør $b$. \\
            Hvis $a$ så gør $b$ ellers gør $c$.
        \end{block}
        \begin{block}{Løkker}
            \begin{itemize}
                \item for i in range(0,10):
                \pause \item for i in lst:
                \pause \item while(point $<$ 100):
            \end{itemize}
    \end{block}
    \end{frame}


    \begin{frame}{Funktions definition}
            \begin{block}{prikprodukt}
                For vektoren $v$ og vektoren $u$ er prikproduktet defineret som
                $$
                    v_1 \cdot u_1 + v_2 \cdot u_2
                $$
                En funktion er en måde at genbruge sin kode på.
            \end{block}
                \pause
            \begin{block}{Demo}
                 Så er der live demo!
            \end{block}
    \end{frame}


    \begin{frame}{Funktions definition}
            \begin{block}{Fibonacci tal}
                Det $n$'te fibonacci tal er defineret som
                $$
                    fib(n-1) + fib(n-2)
                $$
            \end{block}
            \pause
            \begin{block}{Demo}
                 Den koder vi!
            \end{block}
    \end{frame}


    \begin{frame}{Funktions definition}
        \begin{block}{Fakultets funktionen}
            Fakultetfunktionens af $n$ (typisk skrevet $n!$) er defineret som
            $$
                n * (n-1) * (n-2) * \dots * 1
            $$
        \end{block}

        \begin{block}{Demo}
             Den koder vi også!
        \end{block}
    \end{frame}

    \begin{frame}{Hvad med et gætte spil?}
         Vi ønsker at kode et lille spil hvor brugeren skal gætte
         det tal computeren har valgt.

        \begin{block}{Demo}
            Den koder vi også!
        \end{block}
    \end{frame}

    \section{Øvelser}
    \begin{frame}[t]{Øvelsestid}
        Kode øvelser i python!
    \end{frame}
    %--- Next Frame ---%

\end{document}
