\documentclass[12pt,t]{beamer}
\usetheme[greyauthor, % Grå tekst forfatter som KU vil have
         dk, % Sprog
         %style=simple, % Vandmærke eller billede
         footstyle=low, % Fjern stor footer
         wmark, % vandmærke på hver side
         logoplace=left % Logo til venstre
         %,sidebar % makes sidebar
         ]{Frederiksberg}
% nat for Science, ku for generic or unit=ics for DIKU
% Tilføj style=simple for vandmærke
\usepackage{listings} % Pakke til kode
\usepackage{pslatex}        % pæn skrift
\usepackage[utf8]{inputenc} % Implementerer Unicode
\usepackage{algpseudocode}
\usepackage{algorithm}

\title{Fagpakke dag 1}
\subtitle{Algoritmer og problemløsning}
\author{
        Arinbjörn Brandsson \\
        Benjamin Rotendahl  \\
        Mathias Mortensen
}

\date[]{\today}


\begin{document}

\frame[plain]{\titlepage}
 \frame{\tableofcontents}

\section{Programmet}

\begin{frame}
    \frametitle{Program for i dag}
    \begin{block}{Program}
        \begin{itemize}
            \item 09:00-10:30 : Algoritme Oplæg og Øvelser
            \item 10:30-10:45 : Pause
            \item 10:45-12:15 : Programmering Oplæg og Øvelser
            \item 12:15-12:45 : Pause
            \item 12:45-14:30 : Machine Learning Oplæg og Øvelser
            \item 14:30-15:00 : Social Oplæg og Spørgsmål 
        \end{itemize}
    \end{block}
\end{frame}


\section{Introduktion til algoritmer}
    \begin{frame}{Algoritmer}
    \begin{block}{Dette Oplæg vil Dække:}
        \begin{itemize}
            \item Introduktion til algoritmer
            \item Algoritme design og metoder
            \item Hvordan man kan sammenligne forskellige løsninger
            \item Øvelser i algoritmer
        \end{itemize}
    \end{block}
    \end{frame}

    \begin{frame}[c]{Hvad er en Algoritme?}
        \begin{quote}
            An algorithm is a self-contained step-by-step set of operations to
            be performed that can be expressed within a finite amount of space
            and time and in a well-defined formal language.
        \end{quote}
        \pause
        \begin{block}{På dansk}
            En algoritme er en \alert{opskrift} på hvordan et bestemt problem
            kan løses.
        \end{block}
    \end{frame}


    \begin{frame}[plain]{Havregryns algoritme}
        \begin{block}{Eksempel}
        \vspace{-1.5em}
        \begin{algorithm}[H]
            \caption{\newline Indgangsbetingelser: En skål, mælk, havregryn
                     \newline Udgangsbetingelser: Morgenmad
            }
            \begin{algorithmic}
                \While{Skålen ikke er fyldt}
                    \State Hæld Gryn i Skålen
                \EndWhile
                \If{Jeg er tyk}
                    \State mælk = Minimælk
                \Else
                    \State mælk = Letmælk
                \EndIf
                \While{Skålen ikke er fyldt}
                    \State Hæld mælk i Skålen
                \EndWhile
            \end{algorithmic}
        \end{algorithm}
        \end{block}
    \end{frame}

    \begin{frame}
        \frametitle{Krav til en algoritme?}
        \begin{block}{Krav til en algoritme}
        \begin{description}
            \item[Veldefineret] Ingen tvetydigheder og vendinger som:"
            så tager du det bedste resultat $\dots$" \pause
            \item[Terminerer] Den må ikke køre for evigt, du skal garantere
            at den rent faktisk finder sit svar. \pause
            \item[input og output] Jeg skal vide at hvis jeg giver
            den $A$ så returnerer den $B$. \pause
            \item[Kan bevises] Det er muligt både at bevise korrekthed og
            køretid for algoritmen.
        \end{description}
        \end{block}
    \end{frame}

    \begin{frame}[t]{Eksempler på algoritmer}
          Algoritmer bruges inden for alle former for problemløsning. \pause
        \begin{block}{Bioinformatik}
            \emph{Longest commen subsequence: } Sammenligning af DNA strenge
            for at se hvor beslægtet to strenge er. \pause
        \end{block}
        \begin{block}{Primtals faktorisering}
            Bruges i \emph{kryptering: } Basis for at vi kan have sikker
            kommunikation. \pause
        \end{block}

        \begin{block}{Machine Learning}
            En samling af algoritmer der selv kan lære og finde egenskaber
            i store data sæt. Gør det muligt at løse problemer der før var
            uden for menneskers kunnen.
        \end{block}
    \end{frame}


\section{Algoritme vs Algoritme}
    \begin{frame}[t]{Valget af algoritme}
        \begin{exampleblock}{Sorterings algoritmer}
            Givet en liste af $n$ tal ønsker vi at returnere en sorteret liste
            af længde $n$.
        \end{exampleblock}
        \pause
        \vspace{-2em}
        \begin{columns}
            \begin{column}{0.5\textwidth}
                \begin{block}{Hold $A$}
                    \begin{itemize}
                        \onslide<+->{\item Har en computer}
                        \onslide<+->{\item Bruger algoritmen \emph{Merge Sort}}
                        \onslide<+->{\item De kan sortere $10$ millioner tal på
                                     under $20$ minutter}
                        \onslide<+->{\item De kan sortere $100$ millioner tal på
                                    $4$ timer.}
                    \end{itemize}
                \end{block}
            \end{column}
            \begin{column}{0.5\textwidth}
                \begin{block}{Hold $B$}
                    \begin{itemize}[<+->]
                    \item Har en computer der er $1000$ gange hurtigere end
                    hold A
                    \item Bruger algoritmen \emph{Insertion Sort}
                    \item De kan sortere $10$ millioner tal på $5$ timer.
                    \item De kan sortere $100$ millioner tal på \alert{23 dage!}
                    \end{itemize}
                \end{block}
            \end{column}
        \end{columns}
    \end{frame}

    \begin{frame}[t]{Sammenligning af Algoritmer}
        \vspace{-1em}
        \begin{quote}
            Vi bruger begrebet \emph{Køretid} for at beskrive hvordan tiden en
            algoritme bruger stiger med input.
        \end{quote}
        \pause
        \vspace{-1em}
        \begin{block}{Definition på køretid}
            En øvregrænse for den tid der bliver brugt på at løse et problem af
            størelse $n$. Skrives som
            $$
                O(n), O(n^2), O(n \lg n), O(n!), O\left( \frac{a}{b} \right)
            $$
        \end{block}
        \pause
        \begin{exampleblock}{Algoritme for minimums funktionen}
            Givet en liste usorteret $X = [x_1,x_2,\dots,x_n]$ ønsker vi at returnere det
            mindste tal i listen. Hvad er algoritmen og hvad er køretiden?
        \end{exampleblock}
    \end{frame}

    \begin{frame}{Minimums algoritme}
        \begin{block}{Eksempel}
        \vspace{-1.5em}
        \begin{algorithm}[H]
            \caption{\newline Input: En liste $X=[x_1,x_2, \dots, x_n]$
                     \newline Ouput: Det mindste tal i listen.
            }
            \begin{algorithmic}
                \State min = $x_1$
                \For{$x_i$ in X}
                    \If{$x_i < min$}
                        \State min = $x_i$
                    \EndIf
                \EndFor
            \end{algorithmic}
        \end{algorithm}
        \end{block}
        \pause
        \begin{block}{Analyse af algoritmen}
            \centering Køretid? \pause $O(n)$ \\
            \pause
            \centering Er den optimal? \pause \alert{Jeps!}
        \end{block}
    \end{frame}


    \begin{frame}[t]{Eksempler på køretid}
        \vspace{-2em}
        \begin{columns}
            \begin{column}{0.33\textwidth}
                \begin{block}{Bogo Sort}
                    \begin{itemize}
                        \item Køretid på $O(n!)$
                    \end{itemize}
                \end{block}
            \end{column}
            \begin{column}{0.33\textwidth}
                \begin{block}{Insertion Sort}
                    \begin{itemize}
                        \item Køretid? $O(n^2)$
                    \end{itemize}
                \end{block}
            \end{column}

            \begin{column}{0.33\textwidth}
                \begin{block}{Merge sort}
                    \begin{itemize}
                        \item Køretid på $O(n \lg n)$
                    \end{itemize}
                \end{block}
            \end{column}
        \end{columns}
        \begin{figure}[h!]
            \caption{Graf over køretider}
            \centering
            \includegraphics[width=0.7\textwidth]{./include/exs.png}
        \end{figure}
    \end{frame}


\section{Algoritme design}
    \begin{frame}[c]{Hvordan finder man på en algoritme}
        \begin{block}{Algoritme for algoritmer}
            \begin{enumerate}
                \item Beskriv problemet med egne ord. \pause
                \item Del problemet op i mindre dele. \pause
                \item Definer output \pause
                \item Definer input \pause
                \item Beskriv trin for at gå fra input til output
            \end{enumerate}
        \end{block}
    \end{frame}

% Eksempel 2
    \begin{frame}{Fibonacci}
    \begin{exampleblock}{Algoritme for at finde et Fibonacci Tal}
        Givet et tal n, vil vi gerne finde det n'te Fibonacci tal i 
        Fibonacci rækken ($[1, 1, 2, 3, 5, 8, 13, 21, ...]$). For eksempel, 
        hvis vi giver som input $6$, skal algoritmen returnere $13$.
    \end{exampleblock}
    \begin{block}{En løsning}
        En mulig løsning: Divide and Conquer, ved brug af rekursion.\pause\\
        \begin{itemize}
        \item Divide and Conquer: Split problemet op i mindre dele som er nemme at 
              løse, og opsaml problemerne efterfølgende
        \pause
        \item Rekursion: En funktion der på et tidspunkt kalder sig selv igen.
        \end{itemize}
        \pause
    \end{block}
    \end{frame}

    \begin{frame}{Fibonacci (cont.)}
    \begin{block}{Observation:}
    For at finde det n'te Fibonacci tal, skal vi finde det (n-1)'te og (n-2)'te 
    Fibonacci tal
    \end{block}
    \pause
        \begin{block}{Algoritmen}
        \vspace{-1.5em}
        \begin{algorithm}[H]
            \caption{\newline Input: Et tal, n
                     \newline Ouput: Det n'te Fibonacci tal
            }
            \begin{algorithmic}
                \Function{fib}{n}
                    \If{n is 0 or 1}
                        ~\Return 1
                    \Else
                        ~\Return \Call{fib}{n-1} + \Call{fib}{n-2}
                    \EndIf
                \EndFunction
            \end{algorithmic}
        \end{algorithm}
        \end{block}
    \end{frame}

    \begin{frame}{Køretid}
        \begin{block}{Køretiden}
            Vi kan illustrere udregningen af fibonacci tal som et binært træ:
            \pause
            {\center
            \includegraphics[scale=0.5]{include/fib.png}
            }
            \pause

            Et binært træ kan indeholde maksimalt $2^n$ blade, så køretiden er $O(2^n)$.
        \end{block}
    \end{frame}

    \begin{frame}{Effektivitet}
        \begin{block}{Effektivitet}
            \begin{itemize}
            \item Køretiden er på $O(2^n)$. Er dette optimalt? \pause \alert{Nope!} \pause
            \item Den nuværende algoritme laver rigtig mange genberegninger - dette er 
            ikke særlig effektivt. \pause
            \item Ny idé: Vi regner Fibonacci tallene fra start, og gemmer de foregående 
            resultater (Dynamic Programming)
            \end{itemize}
        \end{block}
    \end{frame}

    \begin{frame}{Algoritmen}
        \vspace{-1em}
        \begin{exampleblock}{Algoritmen}
        \vspace{-1.5em}
        \begin{algorithm}[H]
            \caption{\newline Input: Et tal, n
                     \newline Ouput: Det n'te Fibonacci tal
            }
            \begin{algorithmic}
                \Function{fib}{n}
                    \State counter     = 2
                    \State fib1        = 1
                    \State fib2        = 1
                    \While{counter $\leq$ n}
                        \State temp = fib1 + fib2
                        \State fib2 = fib1
                        \State fib1 = temp
                        \State counter = counter + 1
                    \EndWhile\\
                    \Return fib1
                \EndFunction
            \end{algorithmic}
        \end{algorithm}
        \end{exampleblock}
    \end{frame}

    \begin{frame}{Køretid og Effektivitet}
        \begin{block}{Køretid}
            I denne version regner vi fra de første Fibonacci tal op til det 
            n'te. Hvad er vores køretid så? \pause $O(n)$. Er dette optimalt? 
            \pause \alert{Jah!} 
        \end{block}
        \pause
        \begin{block}{Effektivitet}
            Demonstration!
        \end{block}
    \end{frame}

\section{Øvelser}
        \begin{frame}[t]{Øvelser}
            \begin{block}{Algoritme for øvelserne}
                \begin{enumerate}
                    \item Der præsenteres et problem med eksempler. \pause
                    \item I finder på en algoritme for problemet
                          (Arbejd gerne sammen) \pause
                    \item Vi løser den sammen på tavlen.
                \end{enumerate}
            \end{block}
        \end{frame}

    \begin{frame}
      \frametitle{Søgning}
      \begin{block}{Mål}
      Givet en sorteret liste og et element, bestem om element er i listen,
      ved at kigge på så få elementer som muligt.
      \end{block}
      \pause

      \begin{exampleblock}{Eksempel}
      Lad en liste være givet ved $[2, 4, 5, 7, 8, 11, 25]$, hvor vi ønsker at
      finde ud af om elementet 11 er listen. Svaret skulle gerne være ja.
      (det første element har indeks 0).
      \end{exampleblock}
    \end{frame}

    \begin{frame}
      \frametitle{Sortering}
      \begin{block}{Mål}
      Sorter en givet usorteret liste.
      \end{block}
      \pause

      \begin{exampleblock}{Eksempel}
      Lad en liste være givet ved [7, 4, 5, 12, 1], denne vil vi gerne sortere!
      Den sorterede liste skulle gerne være [1, 4, 5, 7, 12].
      \end{exampleblock}
    \end{frame}

    \begin{frame}{Primtal}
      \begin{block}{Mål}
      Skriv en algoritme der finder det $n$'te primtal - prøv at gør den så
      hurtig som mulig.
      \end{block}
      \pause

      \begin{exampleblock}{Eksempel}
      Find det femte primtal. Algoritmen skal gerne returnere 11.
      \end{exampleblock}
    \end{frame}

    \begin{frame}{Korteste Vej}
      \begin{block}{Mål}
        Betragt grafen nedenunder, hvor tallene ved hver kant repræsenterer 
        en distance. Beskriv en algoritme som finder den korteste vej imellem 
        to noder.
      \end{block}
      \includegraphics{shortest.png}
      \begin{exampleblock}{Eksempel}
        Hvis vi gerne vil finde den korteste vej imellem A og F, skal jeres 
        løsning returnere vejen A-$>$B-$>$F, med vægt på 58.
      \end{exampleblock}
    \end{frame}


\section{Opsamling og Spørgsmål}
    \begin{frame}[c]{Spørgsmål}
        \begin{block}{Næste gang}
            I lærer at implementere algoritmer så i kan få en computer til at
            køre dem.
        \end{block}
        \begin{quote}
            \centering Nogle Spørgsmål?
        \end{quote}
    \end{frame}
\end{document}
