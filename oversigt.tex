\documentclass[11pt]{article}
\usepackage[a4paper, hmargin={2.8cm, 2.8cm}, vmargin={2.5cm, 2.5cm}]{geometry}
\usepackage{eso-pic} % \AddToShipoutPicture
\usepackage{graphicx} % \includegraphics

% Pakker til skrifttyper, tekst osv.
% %%%%%%%%%%%%%%%%%%%%%%%%%%%%%%%%%%%%%%%%%%%%%%%%%%%%%%%%%%%%%%%%%%%%%%%%%%%%%
    \usepackage[utf8]{inputenc} % Implementere Unicode
    \usepackage[T1]{fontenc}    % Unicode skrifttype, fx. é skrives som 1 tegn
    \usepackage[english]{babel} % Engelsk Ordbog
  % \usepackage[danish]{babel}  % Dansk Ordbog
    \usepackage{microtype}      % Forbedre linjeombrydningen
    \usepackage{libertine}      % Skrifttype
    \usepackage[scaled=0.83]{inconsolata} % Skrifttype til kode til kode
% %%%%%%%%%%%%%%%%%%%%%%%%%%%%%%%%%%%%%%%%%%%%%%%%%%%%%%%%%%%%%%%%%%%%%%%%%%%%%


% Pakker til matematik og kode.
% %%%%%%%%%%%%%%%%%%%%%%%%%%%%%%%%%%%%%%%%%%%%%%%%%%%%%%%%%%%%%%%%%%%%%%%%%%%%%
    \usepackage{mathtools}       % Udvidelse til amsmath pakken
    \usepackage{algpseudocode}   % pseudocode til algoritmer
    \usepackage{algorithm}       % Pakke til algoritmer
    \usepackage{amsthm}          % Pakke til Theroms
% %%%%%%%%%%%%%%%%%%%%%%%%%%%%%%%%%%%%%%%%%%%%%%%%%%%%%%%%%%%%%%%%%%%%%%%%%%%%%

% Pakker til layout.
% %%%%%%%%%%%%%%%%%%%%%%%%%%%%%%%%%%%%%%%%%%%%%%%%%%%%%%%%%%%%%%%%%%%%%%%%%%%%%
    \usepackage{fancyhdr}        % Gør det muligt at bruge sidehoveder
    \usepackage{graphicx}        % Mulighed for bl.a. \includegraphics
    \usepackage[dvipsnames]{xcolor} % Mulighed for \definecolor
    \usepackage{colortbl}        % Hvis man vil farvelægge sine tabeller
    \usepackage{array}           % Gør miljøerne array og tabular lidt bedre
    \usepackage{parskip}         % Første paragraf i afsnit indrykkes ikke
    \usepackage{listings}        % Pakke til at indsætte kode
    \usepackage{enumitem}        % Gør det muligt at tilpasse lister
    \usepackage{titlesec}        % Tilpassing af afstand mellem sektioner
    \usepackage[lastpage,user]{zref} % Side x af y
% %%%%%%%%%%%%%%%%%%%%%%%%%%%%%%%%%%%%%%%%%%%%%%%%%%%%%%%%%%%%%%%%%%%%%%%%%%%%%


% Ikke standard pakker
% %%%%%%%%%%%%%%%%%%%%%%%%%%%%%%%%%%%%%%%%%%%%%%%%%%%%%%%%%%%%%%%%%%%%%%%%%%%%%
%    \usepackage{boxproof}        % Til at lave bevis kasser i logik
    %\usepackage{semantic}        % Pakke til logik med ->, |- osv.
%    \usepackage{daymonthyear}    % Giver info om dato.
% %%%%%%%%%%%%%%%%%%%%%%%%%%%%%%%%%%%%%%%%%%%%%%%%%%%%%%%%%%%%%%%%%%%%%%%%%%%%%


% Implementere en række makroer og de pakker der er importeret
% %%%%%%%%%%%%%%%%%%%%%%%%%%%%%%%%%%%%%%%%%%%%%%%%%%%%%%%%%%%%%%%%%%%%%%%%%%%%%
    \pagestyle{fancy}                        % Implementere sidehoved
    \lhead{Benjamin Rotendahl} % Venstre sidehoved
    \rhead{Gymnasie Tjenseste}                      % Højre sidehoved
    \cfoot{\thepage\ of \zpageref{LastPage}} % Side x af y
    \newtheorem*{prp}{Propostion}            % Skaber nyt theorem
    \setlist{nolistsep}                      % Formindsker mellemrum mellem listepunkter


    % Definitioner af farver
    % %%%%%%%%%%%%%%%%%%%%%%%%%%%%%%%%%%%%%%%%%%%%%%%%%%%%%%%%%%%%%%%%%%%%
        \definecolor{KURed1}{RGB}{144,26,30}    % Official KU Red 1
        \definecolor{KURed2}{RGB}{199,36,41}    % Unofficial KU Red
        \definecolor{KUGray1}{RGB}{102,102,102} % Official KU Gray 1
        \definecolor{KUGray2}{RGB}{133,133,133} % Official KU Gray 2
        \definecolor{KUGray3}{RGB}{163,163,163} % Official KU Gray 3
        \definecolor{KUGray4}{RGB}{194,194,194} % Official KU Gray 4
        \definecolor{KUGray5}{RGB}{224,224,224} % Official KU Gray 5
    % %%%%%%%%%%%%%%%%%%%%%%%%%%%%%%%%%%%%%%%%%%%%%%%%%%%%%%%%%%%%%%%%%%%%


    % Mindsker afstanden mellem sektioner
    % %%%%%%%%%%%%%%%%%%%%%%%%%%%%%%%%%%%%%%%%%%%%%%%%%%%%%%%%%%%%%%%%%%%%
        \titlespacing\section{0pt}{12pt plus 4pt minus 2pt}
                                  {0pt plus 1pt minus 3pt}
        \titlespacing\subsection{0pt}{12pt plus 4pt minus 2pt}
                                  {0pt plus 1pt minus 3pt}
        \titlespacing\subsubsection{0pt}{12pt plus 4pt minus 2pt}
                                  {0pt plus 1pt minus 3pt}
    % %%%%%%%%%%%%%%%%%%%%%%%%%%%%%%%%%%%%%%%%%%%%%%%%%%%%%%%%%%%%%%%%%%%%


    % Genveje
    % %%%%%%%%%%%%%%%%%%%%%%%%%%%%%%%%%%%%%%%%%%%%%%%%%%%%%%%%%%%%%%%%%%%%
    \def\meta#1{\mbox{$\langle\hbox{#1}\rangle$}} % Mbox til premiser
    \def\intro#1{{#1}{\cal I}}
    \def\elim#1{{#1}{\cal E}}
    \let\imp\to
    \let\implies\to

    % %%%%%%%%%%%%%%%%%%%%%%%%%%%%%%%%%%%%%%%%%%%%%%%%%%%%%%%%%%%%%%%%%%%%


    % Milø til kode
    % %%%%%%%%%%%%%%%%%%%%%%%%%%%%%%%%%%%%%%%%%%%%%%%%%%%%%%%%%%%%%%%%%%%%
        \lstdefinestyle{kode}{
            basicstyle=\scriptsize\ttfamily\color{black},
            keywordstyle=\bfseries\color{KURed1},
            commentstyle=\bfseries\color{KUGray1},
            identifierstyle=\color{black},
            stringstyle=\color{KURed2},
            breaklines=false,
            showspaces=false,
            showstringspaces=false,
            extendedchars=true,
            breakatwhitespace=false,
        }
        \lstnewenvironment{code}[1][]
        {\minipage{\linewidth}
            \lstset{
                #1,
                style=kode,
                escapeinside={!!}{!!},
                frame=tb,
                }
        }
        {\endminipage}
    % %%%%%%%%%%%%%%%%%%%%%%%%%%%%%%%%%%%%%%%%%%%%%%%%%%%%%%%%%%%%%%%%%%%%


    % Laver titel
    % %%%%%%%%%%%%%%%%%%%%%%%%%%%%%%%%%%%%%%%%%%%%%%%%%%%%%%%%%%%%%%%%%%%%
    \title{
      \vspace{13em}
      \Large{Københavns Universitet} \\
      \Huge{3-Ugers forløb i Data analyse}
    }

    \author{
      \Large{Benjamin Rotendahl - XMT224}
      \\ \texttt{Bero@di.ku.dk} \\
    }

    \date{
        \vspace{22em}
        \today
    }
    % %%%%%%%%%%%%%%%%%%%%%%%%%%%%%%%%%%%%%%%%%%%%%%%%%%%%%%%%%%%%%%%%%%%%

% %%%%%%%%%%%%%%%%%%%%%%%%%%%%%%%%%%%%%%%%%%%%%%%%%%%%%%%%%%%%%%%%%%%%%%%%%%%%%


%%%%%%%%%%%%%%%%%%%%      Her starter documentet    %%%%%%%%%%%%%%%%%%%%%%%%%%%
\begin{document}


    %% Change `ku-farve` to `nat-farve` to use SCIENCE's old colors or
    %% `natbio-farve` to use SCIENCE's new colors and logo.
    \AddToShipoutPicture*{\put(0,0){\includegraphics*[viewport=0 0 700 600]{include/ku-farve}}}
    \AddToShipoutPicture*{\put(0,602){\includegraphics*[viewport=0 600 700 1600]{include/natbio-farve}}}

    %% Change `ku-en` to `nat-en` to use the `Faculty of Science` header
    \AddToShipoutPicture*{\put(0,0){\includegraphics*{include/nat-en}}}

    \clearpage

%Disse linjer skaber forside, evt indholdsfortegnelse, og sætter sidetal
%%%%%%%%%%%%%%%%%%%%%%%%%%%%%%%%%%%%%%%%%%%%%%%%%%%%%%%%%%%%%%%%%%%%%%%%%%%%%
    \maketitle              % Forside
    \thispagestyle{empty}   % Fjerner sidetal forside

        % Slå disse til hvis der ønskes indholdsfortegnelse
        % %%%%%%%%%%%%%%%%%%%%%%%%%%%%%%%%%%%%%%%%%%%%%%%%%%%%%%%%%%%%%%%%%%%%%
            %\newpage                % Side til indholdsfortegnelse
            %\thispagestyle{empty}   % Fjerner sidetal fra indholdsfortegnelse
            %\tableofcontents        % Skaber indholdsfortegnelse
        % %%%%%%%%%%%%%%%%%%%%%%%%%%%%%%%%%%%%%%%%%%%%%%%%%%%%%%%%%%%%%%%%%%%%%

    \newpage                % Første rigtige side
    \setcounter{page}{1}    % Sætter rigtigt sidetal på første side
%%%%%%%%%%%%%%%%%%%%%%%%%%%%%%%%%%%%%%%%%%%%%%%%%%%%%%%%%%%%%%%%%%%%%%%%%%%%%

\section{Introduktion}
    Dette dokument stammer fra gymnasie tjensten fra
    Datalogisk Institut ved København Universitets (DIKU)

    Gymnasie tjenesten formål er at få flere studerende til at vælge en IT-uddannelse,
    eller en kombinationsuddannelse af IT og et andet fag og
    skabe en interesse for datalogi.

    Vores plan for at nå dette mål er at udvikle udervisningsforløber til gymnasie klasser
    der giver en introduktion til datalogi. Disse forløb vil vi i gymnasie tjenesten rejse
    rundt og gennemføre med eleverne i undervisningstimerne.

    Vi estimerer at sådan et forløb ville tage cikra 6 skoletimer, der afvikles over 3 uger,
    med to timers undervisnings per uge.

    Forløbet ekisterer i to versioner, et til klasser med en form for IT-fag,
    og et forløb til klasser uden IT-fag.

    Vi er stadig i udviklingsfasen af disse forløb, og sætter derfor stor pris på respons
    fra gymnasielærer.

\section{Forløbet for ITfag. }


\section{Forløbet for kombinationsfag}
    I tilfælde af kombinationsfag kan vi ikke bruge den samme tilgang som
    til ITfag grundet manglende erfaring indenfor programmering. Vi vil derfor
    give kombinationsfagene mulighed for at udnytte deres egen data fra et tidligere
    eller igangværende projekt til data analyse.
    Hvis eleverne ikke har data til rådighed, stiller vi data med data eksemempler inden
    for deres studieretningsfag.

    Forløbet er
    \begin{enumerate}
        \item \unnderline{uge - Introduktion til data analyse} ~ \\
        Fremvisning af de forskellige tekniker, 'sjov' demostration og forklaring af den
        algortime der passer bedst til deres forsøgsdata.
        Eleverene arbejder med ``python notebooks'' som er et modificertbart
        undervisningsværktøj til python der ikke kræver installation, hvori der gennemgåes
        øvelser i data analyse


        \item \unnderline{uge - Introduktion til python og programmering} ~ \\
        Der gennemgåes en python en python notebook hvor eleverne introduceres til
        basal programmering, (variabler/data-typer, løkker, if/else betingelser, funktioner)


        \item \unnderline{uge - Kombination af data analyse og python} ~ \\
        Eleverne skal nu sammensætte deres nye viden inden for data analyse og programmering
        og selv implementere en algortime hvor vi har givet en skabalon der indeholder
        programmets struktur.
    \end{enumerate}







\end{document}
