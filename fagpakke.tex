\documentclass[11pt]{article}
\usepackage[a4paper, hmargin={2.8cm, 2.8cm}, vmargin={2.5cm, 2.5cm}]{geometry}
\usepackage{eso-pic} % \AddToShipoutPicture
\usepackage{graphicx} % \includegraphics

% Pakker til skrifttyper, tekst osv.
% %%%%%%%%%%%%%%%%%%%%%%%%%%%%%%%%%%%%%%%%%%%%%%%%%%%%%%%%%%%%%%%%%%%%%%%%%%%%%
    \usepackage[utf8]{inputenc} % Implementere Unicode
    \usepackage[T1]{fontenc}    % Unicode skrifttype, fx. é skrives som 1 tegn
 %    \usepackage[english]{babel} % Engelsk Ordbog
   \usepackage[danish]{babel}  % Dansk Ordbog
    \usepackage{microtype}      % Forbedre linjeombrydningen
    \usepackage{libertine}      % Skrifttype
    \usepackage[scaled=0.83]{inconsolata} % Skrifttype til kode til kode
% %%%%%%%%%%%%%%%%%%%%%%%%%%%%%%%%%%%%%%%%%%%%%%%%%%%%%%%%%%%%%%%%%%%%%%%%%%%%%


% Pakker til matematik og kode.
% %%%%%%%%%%%%%%%%%%%%%%%%%%%%%%%%%%%%%%%%%%%%%%%%%%%%%%%%%%%%%%%%%%%%%%%%%%%%%
    \usepackage{mathtools}       % Udvidelse til amsmath pakken
    \usepackage{algpseudocode}   % pseudocode til algoritmer
    \usepackage{algorithm}       % Pakke til algoritmer
    \usepackage{amsthm}          % Pakke til Theroms
% %%%%%%%%%%%%%%%%%%%%%%%%%%%%%%%%%%%%%%%%%%%%%%%%%%%%%%%%%%%%%%%%%%%%%%%%%%%%%

% Pakker til layout.
% %%%%%%%%%%%%%%%%%%%%%%%%%%%%%%%%%%%%%%%%%%%%%%%%%%%%%%%%%%%%%%%%%%%%%%%%%%%%%
    \usepackage{fancyhdr}        % Gør det muligt at bruge sidehoveder
    \usepackage{graphicx}        % Mulighed for bl.a. \includegraphics
    \usepackage[dvipsnames]{xcolor} % Mulighed for \definecolor
    \usepackage{colortbl}        % Hvis man vil farvelægge sine tabeller
    \usepackage{array}           % Gør miljøerne array og tabular lidt bedre
    \usepackage{parskip}         % Første paragraf i afsnit indrykkes ikke
    \usepackage{listings}        % Pakke til at indsætte kode
    \usepackage{enumitem}        % Gør det muligt at tilpasse lister
    \usepackage{titlesec}        % Tilpassing af afstand mellem sektioner
    \usepackage[lastpage,user]{zref} % Side x af y
% %%%%%%%%%%%%%%%%%%%%%%%%%%%%%%%%%%%%%%%%%%%%%%%%%%%%%%%%%%%%%%%%%%%%%%%%%%%%%


% Ikke standard pakker
% %%%%%%%%%%%%%%%%%%%%%%%%%%%%%%%%%%%%%%%%%%%%%%%%%%%%%%%%%%%%%%%%%%%%%%%%%%%%%
%    \usepackage{boxproof}        % Til at lave bevis kasser i logik
    %\usepackage{semantic}        % Pakke til logik med ->, |- osv.
%    \usepackage{daymonthyear}    % Giver info om dato.
% %%%%%%%%%%%%%%%%%%%%%%%%%%%%%%%%%%%%%%%%%%%%%%%%%%%%%%%%%%%%%%%%%%%%%%%%%%%%%


% Implementere en række makroer og de pakker der er importeret
% %%%%%%%%%%%%%%%%%%%%%%%%%%%%%%%%%%%%%%%%%%%%%%%%%%%%%%%%%%%%%%%%%%%%%%%%%%%%%
    \pagestyle{fancy}                        % Implementere sidehoved
    \lhead{Københavns Universitet} % Venstre sidehoved
    \rhead{Datalogisk Fagpakke}                      % Højre sidehoved
    \cfoot{\thepage\ of \zpageref{LastPage}} % Side x af y
    \newtheorem*{prp}{Propostion}            % Skaber nyt theorem
    \setlist{nolistsep}                      % Formindsker mellemrum mellem listepunkter


    % Definitioner af farver
    % %%%%%%%%%%%%%%%%%%%%%%%%%%%%%%%%%%%%%%%%%%%%%%%%%%%%%%%%%%%%%%%%%%%%
        \definecolor{KURed1}{RGB}{144,26,30}    % Official KU Red 1
        \definecolor{KURed2}{RGB}{199,36,41}    % Unofficial KU Red
        \definecolor{KUGray1}{RGB}{102,102,102} % Official KU Gray 1
        \definecolor{KUGray2}{RGB}{133,133,133} % Official KU Gray 2
        \definecolor{KUGray3}{RGB}{163,163,163} % Official KU Gray 3
        \definecolor{KUGray4}{RGB}{194,194,194} % Official KU Gray 4
        \definecolor{KUGray5}{RGB}{224,224,224} % Official KU Gray 5
    % %%%%%%%%%%%%%%%%%%%%%%%%%%%%%%%%%%%%%%%%%%%%%%%%%%%%%%%%%%%%%%%%%%%%


    % Mindsker afstanden mellem sektioner
    % %%%%%%%%%%%%%%%%%%%%%%%%%%%%%%%%%%%%%%%%%%%%%%%%%%%%%%%%%%%%%%%%%%%%
        \titlespacing\section{0pt}{12pt plus 4pt minus 2pt}
                                  {0pt plus 1pt minus 3pt}
        \titlespacing\subsection{0pt}{12pt plus 4pt minus 2pt}
                                  {0pt plus 1pt minus 3pt}
        \titlespacing\subsubsection{0pt}{12pt plus 4pt minus 2pt}
                                  {0pt plus 1pt minus 3pt}
    % %%%%%%%%%%%%%%%%%%%%%%%%%%%%%%%%%%%%%%%%%%%%%%%%%%%%%%%%%%%%%%%%%%%%


    % Genveje
    % %%%%%%%%%%%%%%%%%%%%%%%%%%%%%%%%%%%%%%%%%%%%%%%%%%%%%%%%%%%%%%%%%%%%
    \def\meta#1{\mbox{$\langle\hbox{#1}\rangle$}} % Mbox til premiser
    \def\intro#1{{#1}{\cal I}}
    \def\elim#1{{#1}{\cal E}}
    \let\imp\to
    \let\implies\to

    % %%%%%%%%%%%%%%%%%%%%%%%%%%%%%%%%%%%%%%%%%%%%%%%%%%%%%%%%%%%%%%%%%%%%


    % Milø til kode
    % %%%%%%%%%%%%%%%%%%%%%%%%%%%%%%%%%%%%%%%%%%%%%%%%%%%%%%%%%%%%%%%%%%%%
        \lstdefinestyle{kode}{
            basicstyle=\scriptsize\ttfamily\color{black},
            keywordstyle=\bfseries\color{KURed1},
            commentstyle=\bfseries\color{KUGray1},
            identifierstyle=\color{black},
            stringstyle=\color{KURed2},
            breaklines=false,
            showspaces=false,
            showstringspaces=false,
            extendedchars=true,
            breakatwhitespace=false,
        }
        \lstnewenvironment{code}[1][]
        {\minipage{\linewidth}
            \lstset{
                #1,
                style=kode,
                escapeinside={!!}{!!},
                frame=tb,
                }
        }
        {\endminipage}
    % %%%%%%%%%%%%%%%%%%%%%%%%%%%%%%%%%%%%%%%%%%%%%%%%%%%%%%%%%%%%%%%%%%%%


    % Laver titel
    % %%%%%%%%%%%%%%%%%%%%%%%%%%%%%%%%%%%%%%%%%%%%%%%%%%%%%%%%%%%%%%%%%%%%
    \title{
      \vspace{10em}
      \Large{Københavns Universitet} \\
      \Huge{3-Ugers forløb i Data analyse}
    }

    \author{
    	 \Large{Arinbjörn Brandsson - Arbr@di.ku.dk }\\
      	\Large{Benjamin Rotendahl - Bero@di.ku.dk}\\
         \Large{Mathias Fleig Mortensen - Mamo@di.ku.dk }
    }

    \date{
        \vspace{28em}
        \today
    }
    % %%%%%%%%%%%%%%%%%%%%%%%%%%%%%%%%%%%%%%%%%%%%%%%%%%%%%%%%%%%%%%%%%%%%

% %%%%%%%%%%%%%%%%%%%%%%%%%%%%%%%%%%%%%%%%%%%%%%%%%%%%%%%%%%%%%%%%%%%%%%%%%%%%%


%%%%%%%%%%%%%%%%%%%%      Her starter documentet    %%%%%%%%%%%%%%%%%%%%%%%%%%%
\begin{document}


    %% Change `ku-farve` to `nat-farve` to use SCIENCE's old colors or
    %% `natbio-farve` to use SCIENCE's new colors and logo.
    \AddToShipoutPicture*{\put(0,0){\includegraphics*[viewport=0 0 700 600]{include/ku-farve}}}
    \AddToShipoutPicture*{\put(0,602){\includegraphics*[viewport=0 600 700 1600]{include/natbio-farve}}}

    %% Change `ku-en` to `nat-en` to use the `Faculty of Science` header
    \AddToShipoutPicture*{\put(0,0){\includegraphics*{include/nat-en}}}

    \clearpage

%Disse linjer skaber forside, evt indholdsfortegnelse, og sætter sidetal
%%%%%%%%%%%%%%%%%%%%%%%%%%%%%%%%%%%%%%%%%%%%%%%%%%%%%%%%%%%%%%%%%%%%%%%%%%%%%
    \maketitle              % Forside
    \thispagestyle{empty}   % Fjerner sidetal forside

        % Slå disse til hvis der ønskes indholdsfortegnelse
        % %%%%%%%%%%%%%%%%%%%%%%%%%%%%%%%%%%%%%%%%%%%%%%%%%%%%%%%%%%%%%%%%%%%%%
            %\newpage                % Side til indholdsfortegnelse
            %\thispagestyle{empty}   % Fjerner sidetal fra indholdsfortegnelse
            %\tableofcontents        % Skaber indholdsfortegnelse
        % %%%%%%%%%%%%%%%%%%%%%%%%%%%%%%%%%%%%%%%%%%%%%%%%%%%%%%%%%%%%%%%%%%%%%

    \newpage                % Første rigtige side
    \setcounter{page}{1}    % Sætter rigtigt sidetal på første side
%%%%%%%%%%%%%%%%%%%%%%%%%%%%%%%%%%%%%%%%%%%%%%%%%%%%%%%%%%%%%%%%%%%%%%%%%%%%%

\section{Introduktion}
    Vi repræsenterer gymnasietjenesten fra
    Datalogisk Institut ved København Universitet (DIKU)

    Gymnasietjenestens formål er at få flere studerende til at vælge en
    IT-uddannelse, eller en kombinationsuddannelse af IT sammen med et andet fag
    og skabe en interesse for datalogi.

    Vores plan for at nå dette mål er at udvikle undervisningsforløb til
    gymnasieklasser der giver en introduktion til datalogi. Disse forløb vil
    gymnasietjenesten tilbyde at rejse rundt med og overtage undervisningen i et
    par timer.

    Vi estimerer at sådan et forløb ville tage cirka 6 skoletimer, der afvikles
    over 3 uger, med to timers undervisning per uge. Denne tidsgrænse er dog
    meget åben og kan tilpasses til hvert forløb.

    Forløbet eksisterer i to versioner: et til klasser med en form for IT-fag,
    og et til klasser uden.

    Vi er stadig i udviklingsfasen af disse forløb, og vil derfor sætte stor
    pris på din feedback.

\section{Forløb for IT-klasser}
    Vi har haft svært ved at finde en undervisningsbeskrivelse af diverse
    programmerings fag i gymnasiet. Dette betyder at vi er stærkt interesseret i
    at finde niveauet for f.eks en HTX klasse med programmering som fag så vi
    kan tilrettelægge vores forløb efter deres evner.
    \begin{enumerate}
        \item \underline{Uge - Introduktion til data analyse} ~ \\
        Fremvisning af de forskellige teknikker, 'sjov' demonstration og
        forklaring af den algoritme der passer bedst til deres forsøgsdata.

        \item \underline{Uge - Introduktion til python og algortimer} ~ \\
        Vi laver nogle python opgaver der skal løses for at gøre eleverne
        komfortable med sproget python, samtidigt med at vi giver en
        introduktion til algoritmer.

        \item \underline{Uge - Kombination af data analyse og python via
        Machine Learning} ~ \\
        I denne uge skal eleverne kombinere deres viden fra de forgående uger
        og lave et program der benytter sig af machine learning til at forudsige
        nye data.
    \end{enumerate}


    \subsection*{Uge 1}
        I den første uge vil vi introducere eleverne til forskellige teknikker
        indenfor naturvidenskab til at analysere data. Til dette formål kan der
        enten blive tilsendt data til os som de studerende har arbejdet med
        forinden (for eksempel i forbindelse med et eksperiment) som vi kan
        undervise de studerende i hvordan man kan få nyt information ud af,
        eller vi kan selv komme med noget data som eleverne kan arbejde med.
        Her vil vi gå i dybden med én teknik (for eksempel K-means neighbor)
        som har speciel relevans for den pågældende data. Her vil vi også
        forklare hvad en algoritme er, samt give en forklaring på hvordan den
        pågældende algoritme ser ud og fungerer. Til slut vil vi give de
        studerende nogle små øvelser i ``python notebooks'' som demonstrerer
        hvordan teknikkerne fungerer.

    \subsection*{Uge 2}
        I uge 2 vil vi lave python øvelser med eleverne, vi forstiller os at de
        allerede har evner inden for basal programmering
        (variabler, data-typer, løkker, if/else betingelser, funktioner).
        Disse øvelser giver en intuitiv forståelse af algoritmik og køretid, f.eks
        forskellen på lineær søgning og binær søgning, implementering af
        sorterings algoritmer.

    \subsection{Uge 3}
        I den sidste uge vil vi lade de studerende demonstrere deres nye
        viden ved at implementere den data analyse teknik som vi snakkede
        om i den første uge via machine learning. Vi vil give dem en skabelon
        som viser hvad strukturen af deres endelige kode skal se ud, med
        blanke linjer som de skal udfylde. Her vil vi tilbyde nogle forskellige
        sværhedsgrader (nem, middel og svær) hvor den nemme er for det
        meste fyldt ud i vores skabelon, mens den sværeste sværhedsgrad
        har mange manglende linjer kode som eleven skal udfylde. De
        studerende der hurtigt klarer sig igennem dette kan derefter arbejde
        på nogle mindre opgaver som vi har forberedt.

\section{Forløb for ikke IT-klasser}
 	Vi vil give eleverne en introduktion til data analyse som datalogisk
	disciplin, og vise hvordan programmering kan bruges til at behandle
	data.

    	Forløbet er
    	\begin{enumerate}
        		\item \underline{Uge - Introduktion til data analyse} ~ \\
        		Fremvisning af de forskellige tekniker, 'sjov' demonstration
        		og forklaring af den algoritme der passer bedst til det
        		valgte data.

        		\item \underline{Uge - Introduktion til python og programmering} ~ \\
        		Der gennemgåes en introduktion til python hvor eleverne
        		lærer basal programmering
        		(variabler, data-typer, løkker, if/else betingelser, funktioner)

		\item \underline{Uge - Kombination af data analyse og python} ~ \\
        		Eleverne skal nu sammensætte deres nye viden inden for data
        		analyse og programmering og selv implementere en algoritme
        		hvor vi har givet en skabelon der hjælper eleven på vej.
    \end{enumerate}

    \subsection*{Uge 1}
        I den første uge vil vi introducere eleverne til forskellige teknikker
        indenfor naturvidenskab til at analysere data. Til dette formål kan der
        enten blive tilsendt data til os som de studerende har arbejdet med forinden
        (for eksempel i forbindelse med et eksperiment) som vi kan undervise de
        studerende i hvordan man kan få nyt information ud af,
        eller vi kan selv komme med noget data som eleverne kan arbejde med.
        Her vil vi gå i dybden med én teknik (for eksempel K-means neighbor)
        som har speciel relevans for den pågældende data. Her vil vi også forklare
        hvad en algoritme er, samt give en forklaring på hvordan den
        pågældende algoritme ser ud og fungerer. Til slut vil vi give de
        studerende nogle små øvelser i ``python notebooks'' som demonstrerer
        hvordan teknikkerne fungerer.

    \subsection*{Uge 2}
        I den anden uge vil vi introducere eleverne til basal programmering. Da
        vi regner med at de studerende ikke har prøvet at programmere før
        vil vi undervise dem i python, som er et nemt og tilgivende
        programmeringssprog for nybegyndere. Vi vil introducere dem til hvad
        variabler, data-typer, løkker, betingelser og funktioner er, og give
        dem små opgaver designet til at få dem langsomt introduceret til
        dette. Målet ved denne uge er ikke at gøre dem til
        programmører, men blot til at give dem en idé om hvad programmering er,
        samt hvad det kan bruges til.

    \subsection*{Uge 3}
        I den sidste uge vil vi lade de studerende demonstrere deres nye viden
        ved at implementere den data analyse teknik som vi snakkede om i den
        første uge. Vi vil give dem en skabelon som viser hvad strukturen af
        deres endelige kode skal se ud, med blanke linjer som de skal udfylde.
        Her vil vi tilbyde nogle forskellige sværhedsgrader (nem, middel og svær)
        hvor den nemme er for det meste fyldt ud i vores skabelon, mens den
        sværeste sværhedsgrad har mange manglende linjer kode som eleven skal
        udfylde. De studerende der hurtigt klarer sig igennem dette kan derefter
        arbejde på nogle mindre opgaver som vi har tilberedt.


\end{document}
